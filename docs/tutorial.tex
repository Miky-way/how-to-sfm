\documentclass{book}
\begin{document}
    \chapter{Prerequisites}
        In this chapter, we will discuss essential building blocks needed for SfM (Structure from Motion), or generally speaking, 3D vision.
        In particular, we will first discuss basic geometric transformations which are widely used in computer graphics. We will see visual
        effects on geometric entities (triangle, circle etc), when transformed through some basic geometric transformations. Afterwards, we 
        will assemble transformations in a particular way to form a very basic camera model -- a set of geometric principles that governs the
        process of projecting 3D world points onto 2D image plane. Lastly, we will discuss some miscellaneous topics that will be used in 
        our final pipeline. 

        \section{Transformations}
            \section{2D Transformations}
                \subsubsection{Scaling}

                \subsubsection{Translation}

                \subsubsection{Rotation}

            \subsection{Homogeneous Coordinates: Vectorizing the Transformations}
                \subsubsection{Graphical Intepertation}

                \subsubsection{Limitation: Scale Ambiguity}

            \subsection{3D Transformations}
                \subsubsection{Translation}

                \subsubsection{Translation}

        \section{Camera Models}
            \subsection{Intrinsics Parameters}

            \subsection{Extrinsics Parameters}

            \subsection{Putting it Together}

        \section{Miscellaneous}
            \subsection{Random Sample Consensus (RANSAC)}
        
            \subsection{Reviewing the SVD (Singular Value Decomposition)}

    \chapter{Epipolar Geometry} 
        \section{Fundamental Matrix}

        \section{Pose Estimation}

    \chapter{3D Scene Estimations}
        \section{Triangulation}
            \subsection{Linear}
            \subsection{Nonlinear}

        \section{Perspective-n-Point Algorithm}

    \chapter{Putting It All Together}
\end{document}