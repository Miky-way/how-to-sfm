\documentclass{book}

\usepackage{amsmath}% http://ctan.org/pkg/amsmath
\usepackage{graphicx}
\usepackage{subfig}
\graphicspath{ {figs/} }
\usepackage[margin=1in]{geometry}

\begin{document}
    \chapter{Prerequisites}
        In this chapter, we will discuss essential building blocks needed for SfM (Structure from Motion), or generally speaking, 3D vision.
        In particular, we will first discuss basic geometric transformations which are widely used in computer graphics. We will see visual
        effects on geometric entities (triangle, circle etc), when transformed through some basic geometric transformations. Afterwards, we 
        will assemble transformations in a particular way to form a very basic camera model -- a set of geometric principles that governs the
        process of projecting 3D world points onto 2D image plane. Lastly, we will discuss some miscellaneous topics that will be used in 
        our final pipeline. 

        \section{Transformations} \label{sec_2dtrans}
            
            \subsection{2D Transformations} 
                Given a point $P$ having coordinates $(x,y)^{T}$, basic geometric transformations are as follows:

                \begin{figure}[!ht]%
                    \centering
                    \subfloat[Scaled ($s_{x}=1, s_{y}=2$)]{{\includegraphics[width=4.5cm]{scaled.png} }}%
                    \qquad
                    \subfloat[Translation ($t_{x}=-3, t_{y}=1$)]{{\includegraphics[width=4.5cm]{translated.png} }}%
                    \qquad
                    \subfloat[Rotation ($\theta = 100^{\circ}$)]{{\includegraphics[width=4.5cm]{rotated.png} }}%
                    \caption{Transformations}%
                    \label{fig_transformations}%
                \end{figure}


                \begin{enumerate}
                    \item \textbf{Scaling} is used to either shrink or expand an object graphically, as shown in figure \ref{fig_transformations}(a).
                    Scaling can be obtained by multiplying
                    the original coordinates $x$ and $y$ with $s_{x}$ and $s_{y}$ respectively, as shown in equation below.
                    $s_{x}$ and $s_{y}$ determines the degree of expansion (or compression) in $x$ and $y$ dimension respetively. Scaling
                    factor more than 1 expands the objects while scaling factor below 1 shrinks the objects.
                    
                    \begin{equation} \label{eq_scaleNaive}
                        x' = s_{x} \times x \hspace{10mm} y' = s_{y} \times y
                    \end{equation} 

                    We can rewrite the above equations in terms of matrix vector product, as shown below: 

                    \begin{equation}
                        \begin{bmatrix}
                            x'\\ 
                            y'
                            \end{bmatrix} = \begin{bmatrix}
                            s_{x} & 0\\ 
                            0 & s_{y}
                            \end{bmatrix}
                            \begin{bmatrix}
                            x\\ 
                            y
                            \end{bmatrix}
                    \end{equation}

                    \item \textbf{Translation} displaces an object to another location, as shown in figure \ref{fig_transformations}(b).
                    Transformation can be achieved by adding a translation 
                    factor $t_{x}$ and $t_{y}$ to $x$ and $y$ respectively, as shown in the equation below. $t_{x}$ and $t_{y}$ determines the 
                    direction and magnitude of displacement in $x$ and $y$ coordinates respectively. 

                    \begin{equation} \label{eq_transNaive}
                        x' = x + t_{x} \hspace{10mm} y' = y + t_{y}
                    \end{equation}

                    We can rewrite the above equations in terms of vector algebra, as shown below: 

                    \begin{equation}
                        \begin{bmatrix}
                            x'\\ 
                            y'
                            \end{bmatrix} = 
                            \begin{bmatrix}
                            x\\ 
                            y
                            \end{bmatrix} + \begin{bmatrix}
                            t_{x}\\ 
                            t_{y}
                            \end{bmatrix}
                    \end{equation}
                    \item \textbf{Rotation} rotates an object at particular angle $\theta$ from its origin, as shown in figure \ref{fig_transformations}(c).
                     Mathematically, 
                    the rotated points $x'$ and $y'$ can be obtained using the equation below: 

                    \begin{equation}
                        x' = x cos\theta - ysin\theta \hspace{10mm} y' = xsin\theta + ycos\theta
                    \end{equation}

                    We can rewrite the above equations in terms of matrix vector product, as shown below: 

                    \begin{equation}
                        \begin{bmatrix}
                            x'\\ 
                            y'
                            \end{bmatrix} = \begin{bmatrix}
                            cos\theta & sin\theta\\ 
                            -sin\theta & cos\theta 
                            \end{bmatrix} 
                            \begin{bmatrix}
                            x\\ 
                            y
                            \end{bmatrix}
                    \end{equation}

                    Furthermore, the rotation matrix $R$ must satisfy some properties which are explained below: 
                    \begin{enumerate}
                        \item $R$ must be an orthogonal matrix i.e $R^{-1} = R^{T}$. 
                        \item det $R = \pm 1$ (The rotation matrix $R$ is improper if det $R = -1$, and proper if det $R = +1$)
                    \end{enumerate} 
                \end{enumerate}

            \subsection{Homogeneous Coordinates}
                Given a point $P = (x,y)^{T}$ in cartesian coordinate system, we can append a dummy coordinate to form $P' = (x,y,1)^{T}$
                in another coordinate frame. To go back, we can simply discard the last element. Furthermore, we can generalize the idea;
                given a point $P = (x,y)^{T}$ in cartesian coordinate system, we can use $P' = (kx, ky, k)$ to represent the same point in 
                new coordinate system called \textit{homogeneous coordinates.} Again, to go back, we can divide the $P'=(kx,ky,k)$ with $k$
                to obtain the original point $P$. Graphically, a point $P$ in cartesian coordinate system corresponds to a ray in homogeneous 
                coordinate system going from origin $O = (0,0,0)$ to $(x,y)$ on a plane perpendicular to $z-axis$ which is $k$ units away.
                Using homogeneous coordinates, we can rewrite the transformations we defined in previous 
                section as follows: 
                \begin{enumerate}
                    \item \textbf{Scaling} \begin{equation}
                        k\begin{bmatrix}
                            x'\\ 
                            y'\\ 
                            1
                            \end{bmatrix} = 
                            \begin{bmatrix}
                            s_{x} & 0 & 0\\ 
                            0 & s_{y} & 0\\ 
                            0 & 0 & 1
                            \end{bmatrix}
                            \begin{bmatrix}
                            x\\ 
                            y\\ 
                            1
                            \end{bmatrix}
                    \end{equation}
                    \item \textbf{Translation} \begin{equation}
                        k\begin{bmatrix}
                            x'\\ 
                            y'\\ 
                            1
                            \end{bmatrix} = 
                            \begin{bmatrix}
                            1 & 0 & t_{x}\\ 
                            0 & 1 & t_{y}\\ 
                            0 & 0 & 1
                            \end{bmatrix}
                            \begin{bmatrix}
                            x\\ 
                            y\\ 
                            1
                            \end{bmatrix}
                    \end{equation}
                    \item \textbf{Rotation} \begin{equation}
                        k\begin{bmatrix}
                            x'\\ 
                            y'\\ 
                            1
                            \end{bmatrix} = 
                            \begin{bmatrix}
                            cos\theta & sin\theta & 0\\ 
                            -sin\theta & cos\theta & 0\\ 
                            0 & 0 & 1
                            \end{bmatrix}
                            \begin{bmatrix}
                            x\\ 
                            y\\ 
                            1
                            \end{bmatrix}
                    \end{equation}
                \end{enumerate}
                Using homogeneous coordinates allowed us to represent all transformation to be rewritten in matrix vector product. This property
                will enable us to combine successive transformations in one matrix seamlessly. 

            \subsection{3D Transformations}
                \subsubsection{Translation}

                \subsubsection{Rotation}

        \section{Camera Models}
            \subsection{Intrinsics Parameters}

            \subsection{Extrinsics Parameters}

            \subsection{Putting it Together}

        \section{Miscellaneous}
            \subsection{Random Sample Consensus (RANSAC)}
        
            \subsection{Reviewing the SVD (Singular Value Decomposition)}

    \chapter{Epipolar Geometry} 
        \section{Fundamental Matrix}

        \section{Pose Estimation}

    \chapter{3D Scene Estimations}
        \section{Triangulation}
            \subsection{Linear}

            \subsection{Nonlinear}

        \section{Perspective-n-Point Algorithm}

    \chapter{Putting It All Together}
\end{document}